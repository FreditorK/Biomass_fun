\documentclass[11pt,a4paper]{article}

\usepackage[margin=2.5cm]{geometry}
\usepackage{amsmath}
\usepackage{multirow}
\usepackage{graphicx}
\usepackage{booktabs}
\usepackage{hyperref}
\usepackage{siunitx}
\usepackage{caption}
\usepackage{cleveref}
%\usepackage{apacite}
\setlength {\marginparwidth }{2cm} 
\usepackage{amssymb}
\usepackage{latexsym}
\usepackage{amsfonts}
\usepackage{amsthm}
\usepackage{mathtools}
\usepackage{multirow}
\usepackage{tabularx}
\usepackage{algorithm}
\usepackage{algorithmicx}
\usepackage{algpseudocode}
\usepackage{float}
\usepackage{xcolor}
\usepackage{graphicx}
\usepackage{enumitem}
\usepackage{subcaption}
\usepackage{wrapfig}
\title{Calculating Forest Biomass and Aquatic Ecosystems Extension\\
       Sentinel--2 Analysis for a Yangtze River Delta Study Area}
\author{Kalise Optimal Earth}
\date{November 23rd, 2025}

\begin{document}
\maketitle

\if{\begin{abstract}
    Forests and aquatic ecosystems play a regulatory role in the global carbon and water cycles, but are not frequently monitored in a spatially explicit way.  
    In this hackathon project, we design and implement a workflow to estimate aboveground forest biomass and map surface-water extent and quality for an area in the Yangtze River Delta.  
    Using Sentinel-2 Level-2A imagery accessed via the Copernicus Data Space Ecosystem (CDSE) API, we compute vegetation indices, classify land cover, estimate biomass with simple NDVI-based allometric equations, and detect and characterise water bodies using NDWI, MNDWI, and an NDTI-based turbidity index. 
\end{abstract}}\fi

\section{Introduction}

Forests and inland waters are spatially distributed regulators of
carbon storage, hydrological flow, and biogeochemical cycling.  Forest
aboveground biomass acts as a long-lived carbon pool; when it is
removed by logging or land conversion, stored carbon is emitted as
\(\mathrm{CO_2}\), contributing to climate change.  Rivers, lakes,
wetlands, and riparian zones simultaneously provide water flow
regulation, pollutant filtration, sediment trapping, aquifer recharge
and flood mitigation.  These services are diffuse and dynamic and thus
poorly captured by traditional inventory statistics.

Quantifying the spatial structure of biomass and aquatic ecosystems
is therefore both an applied mathematical problem and a practical
support for conservation and planning.  The aim of this hackathon is
to develop a coherent, fully reproducible workflow that:
(i) estimates forest biomass from multispectral satellite imagery,
(ii) maps the location and type of surface water, and
(iii) assesses basic water-quality patterns.

\begin{wrapfigure}{l}{0.28\textwidth}
  \vspace*{-0.5\baselineskip}
  \includegraphics[width=1\linewidth]{img/area_of_interest.pdf}
  \caption{Area of interest.}
  \vspace*{-\baselineskip}
  \label{fig:area_of_interest}
\end{wrapfigure}

Our AOI (Figure~\ref{fig:area_of_interest}) is a
\SI{25}{\kilo\meter}-radius disc centred at
\((119.8241^\circ,\,31.12085^\circ)\) in the Yangtze River Delta, a
heterogeneous landscape with semi-natural forests, intensive
agriculture, dense settlements and a dense river network.  Sentinel--2
Level--2A imagery provides atmospherically corrected reflectances at
10-20\,m resolution, enabling pixel-level inference of vegetation and
water properties across this mosaic.

\section{Methods}

\paragraph{Vegetation indices.}
We process six spectral bands from Sentinel-2, mosaicking them across tiles and aligning to a common 10-meter grid. 
Digital numbers are converted to reflectance values between $0$ and $1.2$, which accounts for saturated values, while cloud, shadow, and invalid pixels are removed using quality masks. 
Three vegetation indices, NDVI, EVI, and SAVI, are calculated to capture different aspects of vegetation health, using standard parameters to ensure numerical stability and physical plausibility.
These indices provide the basis for both land-cover classification and biomass estimation in later stages. 
The resulting vegetation maps are shown in Figure~\ref{fig:phase2}.

\begin{figure}[!htbp]
    \centering
    \captionsetup[subfigure]{aboveskip=-1pt,belowskip=-1pt}
\begin{subfigure}[t]
  {0.3\textwidth}  \includegraphics[width=\linewidth]{img/phase2_EVI.pdf}
    \subcaption{EVI}
\end{subfigure}
\begin{subfigure}[t]
  {0.3\textwidth}
\includegraphics[width=\linewidth]{img/phase2_NDVI.pdf}
\caption{NVDI}
\end{subfigure}
\begin{subfigure}[t]
  {0.3\textwidth}
\includegraphics[width=\linewidth]{img/phase2_SAVI.pdf}
\caption{SAVI}
\end{subfigure}
\caption{\label{fig:phase2}Vegetation indices.}
\end{figure}

\paragraph{Land cover and biomass model.}
Land cover is classified using NDVI thresholds in accordance with regional guidelines for the Yangtze River Delta.
Using the $10$m pixel size, each pixel represents \(A_\text{pix}=0.01\) ha, enabling future area calculations for the study region. 
The summary table~\ref{tab:land_cover} presents the classification results with four vegetation density classes: {\em non-forest}, {\em sparse}, {\em moderate}, and {\em dense vegetation}.
Aboveground biomass is estimated using the NDVI-based allometric equation from Zheng et al. (2004), calibrated for Asian subtropical forests. 
The relationship $\text{Biomass} = 12.3 + 285.5 \times \text{NDVI}$ provides initial estimates, which are then scaled by vegetation density class to better reflect regional conditions in the Yangtze River Delta. 
Biomass calculations are restricted to vegetated pixels within the area of interest, with values bounded between $0$ and $400$~ton per hectare to maintain ecological plausibility.
Statistical validation included comprehensive sensitivity analysis testing key parameters: NDVI vegetation thresholds ($0.15$-$0.25$) and maximum biomass limits for sparse vegetation ($40$-$60$ t/ha). 
Results were compared against published values for regional ecosystem types, and uncertainty was quantified through confidence scoring based on the proportion of comparisons falling within expected ranges. 
Distribution characteristics, including skewness, coefficient of variation, and outlier percentages, were systematically evaluated to assess result robustness.

\begin{table}[!htbp]
    \centering
    \begin{tabular}{l c c c c c}
    \toprule
    & Non-forest & Sparse & Moderate & Dense & Total \\
    \midrule
    Area (ha) & 70085.9 & 43681.9 & 80601.0 & 1604.2 & 194368.8 \\
    Percentage (\%) & 36.1\% & 22.5\% & 41.5\% & 0.8\% & 100.0\% \\
    \toprule     
    \end{tabular}
    \caption{Land cover classification area statistics showing the vegetation density distribution.\label{tab:land_cover}}
\end{table}

\paragraph{Water detection and quality.}
Water detection employed a dual-index approach calculating Normalized Difference Water Index (NDWI) from green and near-infrared bands and Modified Normalized Difference Water Index (MNDWI) from green and shortwave infrared bands. 
We descriminated water using the standard thresholds NDWI $> 0$ and MNDWI $> 0$ with both individual and combined approaches to minimize false positives. 
Water bodies were classified into permanent, seasonal, and wetland categories based on agreement between indices and NDVI constraints (0.1-0.4) for wetland identification. 
The water quality assessment used the Normalised Difference Turbidity Index (NDTI) from red and green bands, with turbidity classified into four levels (clear, moderate, high, very high) using established thresholds ($-0.1$, $0.0$, $0.1$). 
Water quality maps were generated showing spatial turbidity patterns, and degraded versus healthy water bodies were identified by detecting areas with very high turbidity (NDTI $\ge 0.1$) across different water types, with suspended sediment concentration estimated using combined red and near-infrared reflectance.

\paragraph{Riparian buffer and conservation analysis.}
The riparian buffers were delineated at $30$, $100$, and $300$-meter distances from water bodies to analyze vegetation quality gradients using NDVI. 
Priority conservation areas were identified using a weighted composite of vegetation health ($40$\%), water proximity ($30$\%) and biomass value ($30$\%), with the highest scores $20$\% designated as high priority.

\paragraph{Ecosystem service quantification and valuation.}
Ecosystem services were quantified through biophysical modeling integrating vegetation density, water distribution, and riparian buffers. 
Economic valuation employed benefit transfer with base values from {\bf literature meta-analyses}, adjusted for regional conditions (quality: $0.8$, scarcity: $1.2$, benefit: $1.5$). 
Multi-service hotspots were identified by normalising and summing service layers, with thresholds set at high quantiles.

\section{Results and Analysis}

\paragraph{Vegetation and biomass.}
The NDVI distribution across the area of interest (mean \(0.253\), standard deviation \(0.26\), and range from \(-0.998\) to \(0.83\)) indicates a landscape with substantial vegetation cover, though dominated by moderate rather than dense forest types. 
Large water bodies and urban areas occupy significant portions of the region, while dense forest remains exceptionally scarce at only $0.8$\% of the total area.
Biomass estimation reveals a total aboveground stock of 
\[
B_\mathrm{tot} = 16\,837\,848~\text{t} \approx 16.838~\text{Mt}
\]
distributed across \(125\,876.4\) hectares of vegetated land. 
The biomass density distribution is relatively compact (mean \(133.8\)~t/ha, median \(139.3\)~t/ha, standard deviation \(27.9\)~t/ha), with most values clustered between \(115.0\) and \(154.4\)~t/ha (25th-75th percentiles). 
This tight distribution suggests homogeneous biomass conditions across the vegetated landscape.

\begin{table}[htbp]
  \centering
  \begin{tabular}{lrrr}
    \toprule
    Land-cover class & Total biomass (t) & Mean (t/ha) & Max (t/ha) \\
    \midrule
    Sparse     &  4\,436\,449      & 101.6   & 126.5   \\
    Moderate   & 12\,094\,983      & 150.1   & 183.6   \\
    Dense      &    306\,435       & 191.0   & 249.4   \\
    \bottomrule
  \end{tabular}
  \caption{Aboveground biomass by land-cover class.\label{tab:biomass-by-class}}
\end{table}

Moderate vegetation stores the majority of biomass (\(12.09\times 10^6\)~t, \(\sim 72\%\) of total), consistent with its dominance in the land cover classification. 
Despite dense forest covering only \(0.8\%\) of the area, it achieves the highest biomass densities (mean \(191.0\)~t/ha). 
The overall mean biomass density of \(133.8\)~t/ha aligns with expectations for mixed forests, with values following a approximately normal distribution centered around this mean and bounded by a minimum threshold of \(69.4\)~t/ha. 
This pattern reflects the homogeneous nature of the moderate vegetation class that dominates the region's biomass storage.
Validation against regional studies confirms the estimate falls within expected ranges for mixed forests, and sensitivity analysis shows less than $4$\% variation in total biomass across tested parameter thresholds, indicating robust results despite the homogeneous distribution pattern.

\begin{wrapfigure}{r}{0.37\textwidth}
  \vspace*{-0.5\baselineskip}
  \includegraphics[width=1\linewidth]{img/phase8.pdf}
  \caption{Conservation priority. Higher scores are areas of greater conservation importance.}
  \vspace*{-\baselineskip}
  \label{fig:conservation}
\end{wrapfigure}

\paragraph{Water bodies and quality.}
Water covers nearly one-fifth of the study area ($43,695$~ha, or $17.5$\%), with almost all of it classified as permanent water bodies given the large presence of a lake in our AOI (Figure~\ref{fig:water-analysis}a). 
Seasonal water accounts for just under $2,000$~ha, while wetlands are scarce at only $1$~ha.
The water quality tells that the vast majority of the waters (around $98$\%) presents moderate turbidity levels, which is typical for the sediment-rich Yangtze River system. 
Only small areas appear either very clear ($0.5$\%) or quite turbid ($1.6$\%), and severely degraded water is virtually nonexistent (Figure~\ref{fig:water-analysis}c). 
The strong agreement between our two detection methods (Figure~\ref{fig:water-analysis}b) gives us confidence in these results, which suggest generally healthy aquatic conditions across the region.

\begin{figure}[htbp]
  \centering
  \begin{subfigure}[b]{0.32\textwidth}
    \includegraphics[width=\textwidth]{img/phase6_water_detection.pdf}
    \caption{Water body classification}
  \end{subfigure}
  \hfill
  \begin{subfigure}[b]{0.32\textwidth}
    \includegraphics[width=\textwidth]{img/phase6_scatter.pdf}
    \caption{NDWI vs MNDWI correlation}
  \end{subfigure}
  \hfill
  \begin{subfigure}[b]{0.32\textwidth}
    \includegraphics[width=\textwidth]{img/phase7_quality_water.pdf}
    \caption{Water quality classification}
  \end{subfigure}
  \caption{Water analysis shows (a) where different water types are located, (b) how two different detection methods agree on what's water, and (c) how clear the water appears across the region.\label{fig:water-analysis}}
\end{figure}

\paragraph{Riparian buffer analysis.}
This revealed extensive buffered areas totalling $255,898$~ha, with vegetation quality showing a strong negative relationship with water proximity. 
The $30$-meter buffer exhibited degraded conditions (mean NDVI $-0.132$), improving to marginal quality at $100$ meters (mean NDVI $0.006$) and reaching moderate levels at $300$ meters (mean NDVI $0.148$). 
This pattern suggests anthropogenic pressure immediately adjacent to water bodies. Conservation prioritisation identified $49,858$~ha ($19.5$\% of total buffer area) as high priority, while wetland connectivity scored poorly ($0.305$) despite perfect proximity ($1.000$), indicating that while wetlands are physically connected to water bodies, their small size ($1.03$~ha) limit functional ecosystem connectivity.
Spatial analysis of conservation priorities (Figure~\ref{fig:conservation}) reveals concentrated high-value areas along riparian corridors, highlighting specific zones for targeted restoration efforts.

\paragraph{Ecosystem service quantification and valuation.}
Quantification of five ecosystem services revealed total annual benefits of $1.18$ billion, dominated by flood protection ($723$ million) and aquifer recharge ($249$ million). Water purification delivered the highest per-hectare value ($4,320$/ha/year) across $43,695$~ha of water bodies, while sediment control provided $20$ million annually across riparian zones. The average ecosystem service value was $1,567$/ha/year across the active service areas.

Multi-service analysis identified $69,250$~ha of hotspot areas ($20$\% of AOI) concentrating $110$ million in annual benefits. These hotspots occurred in vegetated floodplains and riparian corridors, where multiple services co-occurred at high levels. The spatial concentration highlights the disproportionate importance of specific landscape elements for ecosystem service delivery.

\begin{table}[!htbp]
  \centering
  \begin{tabular}{lrrr}
    \toprule
    Service & Area (ha) & Total Value (USD/yr) & Value per ha (USD/yr) \\
    \midrule
    Flow Regulation & 1 & 2,225 & 2,160 \\
    Purification & 43,695 & 188,762,314 & 4,320 \\
    Sediment Control & 17,712 & 20,403,729 & 1,152 \\
    Aquifer Recharge & 249,289 & 179,488,145 & 720 \\
    Flood Protection & 249,289 & 722,515,456 & 2,898 \\
    \midrule
    Total & 559,986 & 1,111,171,868 & 1,984 \\
    \bottomrule
  \end{tabular}
  \caption{Ecosystem service valuation summary for the Yangtze River Delta study area.\label{tab:ecosystem-valuation}}
\end{table}

The valuation (see Table~\ref{tab:ecosystem-valuation}) reveals that the 559,986 hectares of land providing these services generate a total of over USD 1.1 billion per year. Among the services, Flood Protection is the most significant in terms of total value (USD 722.5 million/yr), which is consistent with a delta region where flood mitigation is a critical function of wetlands and vegetated land. This is followed by Aquifer Recharge and Water Purification, which are also vital services in a densely populated and economically important estuary.

The data shows a clear distinction between the spatial extent and the value intensity of each service. For instance, while Aquifer Recharge occurs across the entire area (249,289 ha), its per-hectare value (USD 720) is lower than that of Purification (USD 4,320/ha), which, despite covering a smaller area, delivers an exceptionally high value per unit of land.

This table underscores the immense economic justification for the conservation and restoration of natural ecosystems in the region. The biomass stocks you previously estimated (16.8 Mt, mean 133.8 t/ha) are the fundamental ecological capital that produces this substantial annual flow of valuable services.

Figure~\ref{fig:phase9} illustrates the quantification of three critical water-related ecosystem services. The spatial analysis of ecosystem services reveals distinct patterns for water regulation. Water purification services show a moderate removal capacity of approximately 2 kg/year of contaminants per unit area. In contrast, sediment control demonstrates a mostly uniform value of around 0.05 tons per year retained along all major bodies of water, suggesting a widespread and stable function provided by riparian vegetation. The most significant variation is observed in aquifer recharge, which is highly dependent on land cover; dense forests facilitate the highest recharge rates of up to 60 mm per year, while this capacity diminishes in regions with sparser vegetation.

\begin{figure}[!htbp]
    \centering
    \captionsetup[subfigure]{aboveskip=-1pt,belowskip=-1pt}
\begin{subfigure}[t]
  {0.3\textwidth}  \includegraphics[width=\linewidth]{img/phase9_purification.pdf}
    \subcaption{Purification levels.}
\end{subfigure}
\begin{subfigure}[t]
  {0.3\textwidth}
\includegraphics[width=\linewidth]{img/phase9_sediment_control.pdf}
\caption{Sediment control.}
\end{subfigure}
\begin{subfigure}[t]
  {0.3\textwidth}
\includegraphics[width=\linewidth]{img/phase9_aquifer_recharge.pdf}
\caption{Aquifer recharge.}
\end{subfigure}
\caption{\label{fig:phase9}Quantification of purification, sediment control, and aquifer recharge.}
\end{figure}

\vspace{-\baselineskip}

\section{Conclusions}

We developed a reproducible workflow to assess the ecosystem, vegetation, and waters, of an area close to the Yangtze River Delta using Sentinel-2 imagery. 

Results indicate 16.8~Mt of aboveground biomass distributed across $125,876$~ha of vegetated land, with extensive permanent water coverage ($43,695$~ha) exhibiting predominantly moderate turbidity levels.
Riparian analysis revealed degraded vegetation conditions proximal to water bodies (30-m buffer NDVI: $-0.132$), improving with distance (300-m buffer NDVI: $0.148$), and identified 49,858~ha of high-priority conservation areas.
Method validation through sensitivity analysis demonstrated parameter robustness ($\pm 4$\% biomass variation), and comparisons with regional studies confirmed result plausibility. 
The workflow presents a quantitative baseline for natural capital assessment, with applications for targeted conservation planning in degraded riparian corridors.

\end{document}
