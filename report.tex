\documentclass[11pt,a4paper]{article}

\usepackage[margin=2.5cm]{geometry}
\usepackage{graphicx}
\usepackage{amsmath,amssymb}
\usepackage{booktabs}
\usepackage{hyperref}
\usepackage{siunitx}
\usepackage{xcolor}

\hypersetup{
  colorlinks=true,
  linkcolor=blue,
  urlcolor=blue,
  citecolor=blue
}

\title{Forest Biomass and Aquatic Ecosystem Services Assessment\\using Sentinel-2 Satellite Data}
\author{Hackathon Team Name}
\date{\today}

\begin{document}

\maketitle

\begin{abstract}
This report presents a preliminary assessment of forest biomass and aquatic ecosystem services for a \SI{25}{\kilo\meter} radius area of interest (AOI) using Sentinel-2 Level-2A satellite imagery and simple biophysical models.
We describe the main data sources and processing steps, summarize the key results in terms of spatial patterns of biomass and ecosystem services, and discuss limitations and possible improvements.
\end{abstract}

\section{Introduction and Objectives}

Forests and aquatic ecosystems provide a wide range of ecosystem services, including carbon storage, water flow regulation, sediment control, aquifer recharge, and flood protection.
Quantifying these services in a consistent and spatially explicit way is essential for supporting conservation planning and climate adaptation strategies.

In this hackathon project, we use Sentinel-2 Level-2A surface reflectance products to estimate aboveground forest biomass and to derive indicators for five hydrological ecosystem services associated with aquatic ecosystems (rivers, wetlands, lakes and riparian zones) within a predefined AOI.
The AOI is defined as a circle of radius \SI{25}{\kilo\meter} centered at given geographic coordinates and intersects multiple Sentinel-2 tiles.

The specific objectives are:
\begin{itemize}
  \item To preprocess Sentinel-2 Level-2A imagery (cloud masking, band extraction, reflectance scaling) and generate vegetation and water indices over the AOI.
  \item To estimate forest aboveground biomass using published allometric equations based on vegetation indices (e.g.\ NDVI, EVI).
  \item To derive spatial indicators for water flow regulation, water purification, sediment control, aquifer recharge and flood protection using simplified biophysical models.
  \item To aggregate these indicators into ecosystem service value estimates using provided valuation coefficients and adjustment factors.
  \item To visualize and interpret spatial patterns of biomass and ecosystem services, highlighting potential conservation or restoration priority areas.
\end{itemize}

\section{Applied Methodology}

\subsection{Data and Study Area}

The analysis is based on Sentinel-2 Level-2A (MSIL2A) products corresponding to the acquisition date specified in the hackathon guidelines.
Because the AOI spans two Sentinel-2 tiles, both products overlapping the AOI were downloaded in SAFE format from the Copernicus Data Space Ecosystem.

The key datasets are:
\begin{itemize}
  \item Sentinel-2 Level-2A reflectance bands: Blue (B02), Green (B03), Red (B04), NIR (B08), SWIR1 (B11), SWIR2 (B12).
  \item Optional auxiliary rasters (if available): digital elevation model (DEM), slope and long-term mean annual precipitation.
  \item Vector geometry of the AOI: a GeoJSON polygon representing the \SI{25}{\kilo\meter} buffer around the central coordinates.
\end{itemize}

\subsection{Preprocessing and Indices}

All processing steps were implemented in Python using \texttt{rasterio}, \texttt{geopandas}, \texttt{numpy} and related libraries within a Jupyter notebook.

The main preprocessing steps are:
\begin{enumerate}
  \item \textbf{SAFE structure handling:} For each Sentinel-2 product, the band files corresponding to the required spatial resolutions were located within the SAFE directory tree.
  \item \textbf{Band extraction:} The relevant bands (B02, B03, B04, B08, B11, B12) were read as scaled integer arrays and converted to top-of-canopy reflectance in the range $[0,1]$ using the documented scale factor (typically $1/10000$).
  \item \textbf{AOI clipping:} Each band was clipped to the AOI geometry.
  Depending on the implementation, this was done either by reading windows intersecting the AOI or by masking the full tile with the AOI polygon.
  \item \textbf{Resampling and alignment:} To ensure consistent pixel alignment when combining bands of different native resolutions (10\,m and 20\,m), all arrays were resampled onto a common 10\,m grid using bilinear resampling.
  \item \textbf{Cloud and no-data masking:} Clouds and invalid pixels were masked using a combination of the Sentinel-2 quality flags (if available) and simple spectral thresholds.
\end{enumerate}

Based on the preprocessed bands, the following indices were computed:
\begin{itemize}
  \item Normalized Difference Vegetation Index (NDVI)
        \[
          \mathrm{NDVI} = \frac{\mathrm{NIR} - \mathrm{Red}}{\mathrm{NIR} + \mathrm{Red}}.
        \]
  \item Enhanced Vegetation Index (EVI), using the coefficients recommended in the project guidelines.
  \item Soil Adjusted Vegetation Index (SAVI) for areas with sparse vegetation.
  \item Normalized Difference Water Index (NDWI) and Modified NDWI (MNDWI) for water detection.
  \item Normalized Difference Turbidity Index (NDTI) for relative turbidity in water bodies.
\end{itemize}

\subsection{Biomass Estimation}

Forest aboveground biomass was estimated using the allometric equations provided in the hackathon document, selecting the equation corresponding to the dominant forest type in the AOI.
In general, biomass per pixel is modeled as a linear or nonlinear function of a vegetation index, for example
\[
  \mathrm{Biomass} = a + b \cdot \mathrm{NDVI} + c \cdot \mathrm{NDVI}^2,
\]
where the coefficients $(a,b,c)$ and the valid NDVI range are taken from the literature.
Biomass estimates were capped at a reasonable upper bound (e.g.\ 500~ton/ha) to avoid unrealistic values.

Total biomass for the AOI was obtained by summing pixel-wise biomass and multiplying by the pixel area in hectares.

\subsection{Ecosystem Service Indicators}

For aquatic and riparian ecosystems, we followed the simplified biophysical formulas provided in the guidelines for five services:
water flow regulation, water purification, sediment control, aquifer recharge and flood protection.
Each service $S$ is represented as a raster indicator combining area, vegetation condition and ancillary factors such as slope or rainfall.
Examples include:
\begin{itemize}
  \item Water flow regulation proportional to water storage capacity, modeled from wetland area, assumed average depth and a vegetation-based retention factor.
  \item Water purification modeled from filtration rate, vegetation quality (NDVI-based) and contact time.
  \item Sediment control estimated from a simplified erosion proxy based on $(1-\mathrm{NDVI})$ and slope, combined with a sediment delivery ratio and trap efficiency.
  \item Aquifer recharge based on precipitation, infiltration rate and runoff coefficient, with NDVI used as a proxy for infiltration capacity.
  \item Flood protection derived from floodplain area, assumed storage depth and a roughness factor linked to vegetation cover.
\end{itemize}

\subsection{Dynamic Valuation}

Biophysical indicators were converted to monetary values using baseline valuation coefficients (USD/ha/year) for each service.
These coefficients were then adjusted by a set of dimensionless factors describing ecosystem condition, water scarcity and downstream population or infrastructure exposure.
The total ecosystem service value for each pixel and for the entire AOI was computed as
\[
  \mathrm{Total\_ESV} = \sum_i \mathrm{Service}_i \times \mathrm{Area}_i \times \mathrm{Quality\_Factor}_i \times \mathrm{Adjustment\_Factors}_i.
\]

\section{Main Results}

\subsection{Forest Biomass}

% Example figure placeholder
\begin{figure}[ht]
  \centering
  % \includegraphics[width=0.8\textwidth]{figures/biomass_map.png}
  \caption{Spatial distribution of estimated aboveground forest biomass (ton/ha) within the AOI.}
  \label{fig:biomass_map}
\end{figure}

A continuous map of forest biomass was generated for the AOI using the selected allometric equation.
Higher biomass values are concentrated in densely forested areas, especially along the upland regions to the west and south of the main lake.
Lower biomass is observed in urban and agricultural zones.

Table~\ref{tab:biomass_summary} summarizes basic statistics of biomass over the AOI.

\begin{table}[ht]
  \centering
  \caption{Summary statistics of aboveground biomass (ton/ha) within the AOI.}
  \label{tab:biomass_summary}
  \begin{tabular}{lrrrr}
    \toprule
    Metric & Min & Mean & Median & Max \\
    \midrule
    Biomass (ton/ha) & -- & -- & -- & -- \\
    \bottomrule
  \end{tabular}
\end{table}

\subsection{Aquatic Ecosystem Services}

% Example figure placeholder
\begin{figure}[ht]
  \centering
  % \includegraphics[width=0.8\textwidth]{figures/water_flow_regulation.png}
  \caption{Example ecosystem service map: water flow regulation capacity (relative units).}
  \label{fig:water_flow}
\end{figure}

Raster maps were produced for each of the five targeted ecosystem services.
In general, high service provision is associated with intact wetlands and vegetated riparian zones, while degraded or urbanized shorelines show lower values.

A concise summary of total and per-hectare values for each service is shown in Table~\ref{tab:esv_summary}.

\begin{table}[ht]
  \centering
  \caption{Illustrative ecosystem service value summary for the AOI (values to be filled in).}
  \label{tab:esv_summary}
  \begin{tabular}{lrr}
    \toprule
    Ecosystem Service & Total value (USD/year) & Mean value (USD/ha/year) \\
    \midrule
    Water flow regulation   & -- & -- \\
    Water purification      & -- & -- \\
    Sediment control        & -- & -- \\
    Aquifer recharge        & -- & -- \\
    Flood protection        & -- & -- \\
    \bottomrule
  \end{tabular}
\end{table}

\section{Analysis and Discussion}

The results highlight strong spatial heterogeneity in both biomass and ecosystem service provision across the AOI.
Forested uplands contribute disproportionately to carbon storage and sediment control, while low-lying wetlands and floodplains around the lake are key for water purification, aquifer recharge and flood attenuation.

Several important points emerge from the analysis:
\begin{itemize}
  \item \textbf{Importance of vegetation condition:} NDVI and related indices strongly influence both biomass and multiple service indicators, underscoring the role of vegetation cover and health.
  \item \textbf{Critical role of riparian zones:} Vegetated buffers along rivers and lake shorelines consistently appear as hotspots of water purification and flood protection services.
  \item \textbf{Data and model limitations:} The biophysical models used are intentionally simple and rely on global default parameters.
  They do not explicitly account for local hydrology, soil properties, management practices or infrastructure, and should be interpreted as relative indicators rather than precise absolute estimates.
  \item \textbf{Uncertainty sources:} Major sources of uncertainty include cloud contamination, misclassification of land cover, limitations of the allometric equations outside their calibration range and the use of global valuation coefficients.
\end{itemize}

Despite these limitations, the workflow demonstrates that freely available satellite data can be combined with simple models to derive integrated maps of biomass and ecosystem services at regional scale.
These outputs can support screening-level assessments and help identify priority areas for more detailed field-based studies.

\section{Conclusions}

This hackathon project implemented an end-to-end workflow for estimating forest biomass and key hydrological ecosystem services using Sentinel-2 Level-2A imagery for a \SI{25}{\kilo\meter} radius AOI.
The methodology integrates satellite-derived vegetation and water indices, literature-based allometric equations and simplified biophysical and economic valuation models.

The main contributions are:
\begin{itemize}
  \item A reproducible Python pipeline for loading, preprocessing and analyzing Sentinel-2 imagery in a geospatial context.
  \item Spatially explicit maps of aboveground forest biomass and five aquatic ecosystem services.
  \item A preliminary valuation of ecosystem services in monetary units, highlighting high-value hotspots within the AOI.
\end{itemize}

Future work could refine the approach by incorporating higher-resolution elevation and hydrological data, locally calibrated biomass and valuation models, and multi-temporal image series to assess seasonal and interannual variability.

\end{document}
